\section{Project Background}

The four of us have been working together as a team since the beginning of our
Year In Industry at Cisco Systems in San Jose, California. During our time there
we gained valuable experience of working with both large and small distributed
computer systems - mainly focused in the world of open-source cloud computing.
In the course of our work we were exposed to various tools for the monitoring of
these distributed systems, but in all cases found them wanting in some respect.
Of the tools with which we came into contact we believe these to be the most
important:

\subsection{Splunk}

Splunk\cite{splunk} is probably the closest to an industry standard when it comes to
analysis of distributed systems. It is a large, enterprise-grade product capable
of solving most of the problems we set out to solve with Apiary. The problem
with Splunk is that it is expensive to use and is largely closed-source. This
does not fit with our philosophy on software transparency and is prohibitive for
those who cannot afford to pay.

\subsection{Elasticsearch ELK}

The Elasticsearch ELK\cite{elk} stack (Elasticsearch, Logstash, and Kibana) is
conceptually very similar to Apiary. It’s open-source and it consists of many
components working together to solve separate parts of the overall problem. A
major difference between Apiary and ELK is in the way that the components
communicate with each other. ELK uses a REST\footnote{Representational state
transfer} API\footnote{Application programming interface} to communicate over HTTP,
whereas Apiary makes use of a distributed message bus. We believe that this
solution is better for scalability and security.
