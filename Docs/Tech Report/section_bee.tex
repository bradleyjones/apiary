\section{Bee}

Bee is our data-harvesting agent, which is installed on all monitored machines
and forwards log data into Hive. We designed Bee to be as lightweight and easy
to set up as possible. This is because we wanted Bee to interfere with the
normal operation of the machine it was monitoring as little as possible, and
for the devops task of actually getting the agents running to be trivial.

For this reason Bee has only 1 configuration option - the IP of your central
RabbitMQ cluster. After that all other configuration is either automatically
discovered from the host machine (MAC address, hostname, etc), or pulled in
from Hive after an initial handshake over the RabbitMQ exchange. After this,
Hive can request that this particular instance of Bee listens to and
forwards changes to any file on that particular machine.

Bee is based on NodeJS\cite{node}, as we needed it to be multiplatform, and the
event-driven nature of its framework lends itself well to file monitoring,
as well as responding to incoming RabbitMQ messages. It is designed using
a Model-View-Controller (MVC) framework, with a series of actions for
different commands from Hive.
