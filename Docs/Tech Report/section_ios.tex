\section{iOS Application}

In the real world a user will not always be sitting in front of their computer
when an important event occurs in the Apiary framework. However, it is likely
that they will be carrying a modern smartphone, which will have a reasonably
consistent WiFi or cellular data connection. Due to our GUI being entirely
web-based a user could access everything they need to know using their
smartphones web browser, but by creating a native application we can enhance the
user experience and provide additional functionality. A small but useful
function is the ability to save the IP address or URL of the particular Queen
instance to hook into, as well as the users login credentials for that instance,
saving a user the time of configuring the application every time they use it.

The greatest argument for a native application is the ability to exploit the
powerful Apple Push Notification System. As described above in the Sting
section, APNS can be used to send the user real-time updates on the status of
their computer system at anytime, as long as they have a cellular/WiFi data
connection. When a user logs into Apiary using the iOS application, their unique
Apple Device ID is registered against their user account to be used by Sting for
alert notifications.
