\section{Future Work}

\subsection{Future Agents}

Very early on in our design we decided that our Agents system should be
flexible enough to accommodate a variety of agent types, as not all schemaless
text data comes from text log files. We had the following Agents planned, but
due to time constraints were not able to implement them, and felt that our time
would be better spent focusing on processing techniques and UI.

First of all, coming from our background with OpenStack, we would have liked to
have built an Agent that was integrated with OpenStack Celiometer, the
monitoring and metrics component of OpenStack. This would have worked in much
the same way, but rather than have the user define a file, they would need to
define a Celiometer Meter to start watching. After this processing would have
worked in exactly the same way.

Another consideration was to build an Agent, or extend Bee so that it could tap
into popular logging frameworks. Many applications have now started to move
away from traditional log files, and instead pump data into logging frameworks
which can maintain slightly more structure about a log. Examples include
LogBack, or Log4j.

Furthermore, other open source projects could be extended to work with our
system, such as FluentD, which has a plugin system that would allow us to build
in Apiary support.

\subsection{Hive - Timemachine and Intelligence}

Some future Hive components we considered but were not within the time scale of
the project are Timemachine and Intelligence. Timemachine would be designed to
pull logs from the database according to a query, and then push them onto a
message queue in the same order that they originally sent. The use case for
this would be an application that requires a replicated stream of historical
data for playing back events as they happened. This historical playback could
be used to analyse system faults to determine what went wrong.

Intelligence is much more complex and has potentially many more use cases. The
idea behind Intelligence is to provide an API for doing Map Reduce tasks with
the log data stored in Honeycomb. Software like Hadoop would allow for more in
depth analysis than is currently possible with the Lucene implementation in
Honeycomb, however both have their place according to the type of query you
need to run. The scope of Intelligence could potentially be extended to include
machine learning which we hope would compliment Pheromone with dynamically
generated alerts.
