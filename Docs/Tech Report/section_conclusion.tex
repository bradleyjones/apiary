\section{Conclusion}

To conclude, we set out to design and build an end-to-end solution for real-time
monitoring and analysis of log data, this we feel we have achieved successfully.
We had several goals which sculpted our design and thought processes for the
project, and we believe we have met them with the solution we have built. At the
beginning of the project we separated our goals into the six major categories
that we felt were most important for a system of this type. Addressing them
individually, we can see how our design decisions helped us achieve each one.

\subsection{Scalability}
Splitting the Hive into several components communicating via RabbitMQ has given
us industry-proven levels of scalability at zero cost and with very little
implementation overhead. The use of MongoDB has also given us a database backend
which has been tested to destruction in all corners of the tech industry.
\subsection{Real Time}
RabbitMQ and web sockets have been used to create truly real time communication
between the many components of Apiary and the web front end. Expensive polling
loops are not required to keep the user up to date with the state of their
system.
\subsection{Simple Configuration}
Designing components such as the Bee to self-configure based on data passed from
the central Hive has made deploying Apiary across a distributed system a much
more straightforward task than many open source projects. The inclusion of
installation scripts has also contributed to this goal significantly.
\subsection{Alert System}
Exploiting the real time nature of our message bus infrastructure and the
recurring query API exposed by Honeycomb has allowed us to create the Pheromone
and Sting components. These components work together with the iOS application to
provide the user with real time feedback regardless of their location.
\subsection{Powerful Query System}
By exploiting the advanced text search language that Lucene provides and
combining this with fielding in Queen we have built a system that allows the
user to filter, and infer schema onto otherwise unorganised data. Lucene's query
language is expansive, and we ourselves are yet to fully explore all of the
features it provides. Data can be searched and fielded on a wide range of
parameters, from content, tag, hostname, log time, and many more.
\subsection{User Friendly UI}
Leveraging the power of our query system, our user interface allows the use to
create rich data visualisations from query results, allowing them to make
valuable and meaningful analysis. Our interface is web based, real-time, and
easy to use.

\vspace{5mm}

Following the points above, we feel that we have have built a valuable tool,
that allows the user to gain real value from data that would otherwise be locked
away, and difficult to processing. Compared to other projects in the market
today we still sit at a relatively primitive stage, simply due to resources,
however we feel that our implementation brings a number of novel design
decisions that give us an edge in certain applications. No other project, for
example employs our system of using message based communication across the whole
stack, which brings a myriad of benefits. It allows us to do real time,  easy
reliable scaling, and deal with heavy loads without scaling. Our fielding system
is also unique, as other systems have opted for expensive, log analysis
algorithms. Our system allows ultimate flexibility, and keeps overheads down.

If we were to start this project again, or were to give guidance to others
untaking similar projects, we would probably try to avoid using NodeJS in
certain parts of our stack. This is simply to to NodeJSs infancy, we found in
a number of cases that it was difficult to debug, due to the lack of tools,
and lack of predictability in the JavaScript specification. Also, we would
spend a little more time planning the communication protocols, as we had to
make a number of changes, at one point switching from XML to JSON. This took a
significant amount time to refactor, and could have been avoided with more
consideration.

Finally, given more more resources, we would have like to have done some
extremely large scale testing. As it stands we were only able to distribute
instances across the machines that we personally owned, which allowed to to
confirm that scaling worked, but did not demonstrate how far it could be
push, in our testing each component was only instanced no more than 3 or 4
times.

