\documentclass[10pt, a4paper, conference, compsocconf]{IEEEtran}

% *** GRAPHICS RELATED PACKAGES ***
%
\ifCLASSINFOpdf
% \usepackage[pdftex]{graphicx}
% declare the path(s) where your graphic files are
% \graphicspath{{../pdf/}{../jpeg/}}
% and their extensions so you won't have to specify these with
% every instance of \includegraphics
% \DeclareGraphicsExtensions{.pdf,.jpeg,.png}
\else
% or other class option (dvipsone, dvipdf, if not using dvips). graphicx
% will default to the driver specified in the system graphics.cfg if no
% driver is specified.
% \usepackage[dvips]{graphicx}
% declare the path(s) where your graphic files are
% \graphicspath{{../eps/}}
% and their extensions so you won't have to specify these with
% every instance of \includegraphics
% \DeclareGraphicsExtensions{.eps}
\fi

% correct bad hyphenation here
\hyphenation{op-tical net-works semi-conduc-tor}


\begin{document}
%
% paper title
% can use linebreaks \\ within to get better formatting as desired
\title{CO600 - Technical Report}


% author names and affiliations
% use a multiple column layout for up to two different
% affiliations

\author{\IEEEauthorblockN{Bradley Jones}
  \IEEEauthorblockA{bj59@kent.ac.uk}
  \and
  \IEEEauthorblockN{Jack Fletcher}
  \IEEEauthorblockA{jpf4@kent.ac.uk}
  \and
  \IEEEauthorblockN{John Davidge}
  \IEEEauthorblockA{jd389@kent.ac.uk}
  \and
  \IEEEauthorblockN{Sam Betts}
  \IEEEauthorblockA{sab50@kent.ac.uk}
}

% make the title area
\maketitle


\begin{abstract}
In this paper we will discuss the motivations for and the development of
Project Apiary - an end-to-end solution for real-time monitoring and analysis
of log data in distributed computer systems. In the development of Project
Apiary we explore the many problems associated with the gathering, storage and
analysis of schema-less data, and seek to define user-friendly approaches for
the presentation of this data. We also explore methods allowing the system to
prompt the user to act on important events as they happen, rather than
discovering them in an after-the-fact analysis.
\end{abstract}

%\begin{IEEEkeywords}
%  component; formatting; style; styling;
%
%\end{IEEEkeywords}

\IEEEpeerreviewmaketitle


\section{Introduction}

Data centres - which are collections of computers connected over a local area
network (LAN) - produce vast quantities of data in the form of application and
machine logs. The larger the data centre and the more complex its
applications, the larger the volume of data produced. With the ongoing
industry moving towards “cloud” computing, data centres are becoming ever
larger and being used for a wider variety of applications, often by thousands
or even tens of thousands of users simultaneously. The logs produced by the
services running on the machines which comprise the data centre contain a wide
range of generic (CPU loading, RAM usage, network traffic, etc) and
application-specific data which can be analysed to gain valuable insight into
the operation of the data centre. However, this data is usually unstructured
and unorganized - making meaningful analysis a non-trivial task. When combined
with the sheer volume of data being produced by ever larger data centers, the
gathering and analysis of this data becomes an ever more difficult task.

Solutions already exist to combat this problem (as detailed in the
background section below), but many of these are closed-source and
expensive, and in our opinion those that are open-source do not currently
tackle every aspect of the problem in a complete end-to-end solution. Given
our background in open-source software we were motivated to explore the
development of an open-source project which covered every component
necessary for a user to gain meaningful insight into the workings of their
data centre.


\section{Project Background}

The four of us have been working together as a team since the beginning of our
Year In Industry at Cisco Systems in San Jose, California. During our time there
we gained valuable experience of working with both large and small distributed
computer systems - mainly focused in the world of open-source cloud computing.
In the course of our work we were exposed to various tools for the monitoring of
these distributed systems, but in all cases found them wanting in some respect.
Of the tools with which we came into contact we believe these to be the most
important:

\subsection{Splunk}

Splunk\cite{splunk} is probably the closest to an industry standard when it comes to
analysis of distributed systems. It is a large, enterprise-grade product capable
of solving most of the problems we set out to solve with Apiary. The problem
with Splunk is that it is expensive to use and is largely closed-source. This
does not fit with our philosophy on software transparency and is prohibitive for
those who cannot afford to pay.

\subsection{Elasticsearch ELK}

The Elasticsearch ELK\cite{elk} stack (Elasticsearch, Logstash, and Kibana) is
conceptually very similar to Apiary. It’s open-source and it consists of many
components working together to solve separate parts of the overall problem. A
major difference between Apiary and ELK is in the way that the components
communicate with each other. ELK uses a REST\footnote{Representational state
transfer} API\footnote{Application programming interface} to communicate over HTTP,
whereas Apiary makes use of a distributed message bus. We believe that this
solution is better for scalability and security.


\section{Aims}

The aim of Apiary was to produce an open source platform for the harvesting,
storage, processing, and visualisation of machine data (logs etc) from large
scale distributed systems, and expose this using a real time, interactive web
front end. In order to ensure that our project was successful, we set out a
series of clear, defined goals before starting the project, as follows.

\begin{itemize}

\item Scalability - During our research we found that typical log rates for
data centre applications could run into terabytes per hour. This meant that we
would need to be able to handle more data than a single appliance could even
store, and still maintain an acceptable level of service. For this reason
building a system that was easily scaled out was of paramount importance.

\item Real Time - We decided quite early on that we would be able to provide a
richer, more interactive service by building a system that operated on a
message based, real time system. This would allow us to push events from the
data center, all the way to the browser, without implementing costly polling
techniques.

\item Simple Configuration - A problem that we identified with a lot of open
source software stacks is that they are often very difficult to configure, and
come with hundreds of configuration options. This can make devops (developer
operations) very difficult, and just finding the optimal set of configurations
can be a task unto itself. For this reason we wanted to keep configuration
options to a minimum, and automatically detect configuration for as much of the
stack as we possibly could.

\item Alert System - One component that we noticed the most of the existing
solutions don't have, is any kind of event alert system. The user can not be
expected to be monitoring the system at all times, so we decided that we should
be able to push alerts directly to the users mobile device.

\item User Friendly UI - Another thing we noticed about a lot of the existing
solutions out there, particularly those that were open source, was that UI was
often treated as a second class citizen to functionality. We wanted to ensure
that our UI was simple, easy to understand, and provided rich data
visualisations that allowed the user to gain the most from their data.

\item Powerful Query System - Our project has a wide range of applications, and
in order to support them all we needed to ensure that the query language could
support a wide range of simple and complex operations, whilst remaining fast
and real time.

\end{itemize}


\section{Architecture and Design}
In this section we will discuss the specific solutions that we built in order to
achieve our goals, the motivations behind our design decisions, and the
technical challenges we overcame. Our project ended up being divided into a
series of independent components; Bee, our data agent; Hive, responsible for
storage and processing of data; Queen, our web front end; and finally, our iOS
application.

The key design decision was to build our system in a ‘Process Oriented’ manner.
This meant all components were broken down into small, scalable single task
components that did not maintain state. All state was stored in a central,
scalable database, and all communication was performed using a scalable
RabbitMQ\cite{rabbit} messaging exchange. Designing the system in this way meant
that we could leverage these technologies to provide a reliable, horizontally
scalable, and real time system.

\graphicspath{{./pics/}}
\begin{figure}[h!]
  \includegraphics[width=8cm, keepaspectratio]{design.jpg}
  \caption{Design of Apiary Architecture}
\end{figure}


The reason for choosing RabbitMQ over other communication paradigms and
technologies, such as REST, is that it provides asynchronous message passing,
with worker queues and publish/subscribe models built in. Worker queues provide
automatic distribution of messages between Apiary components of the same type,
giving Apiary load balancing, and therefore configurationless scaling, for free.
The Publish subscribe model allows Apiary components to subscribe to real time
messages required for live events. It also maintains queues messages in the case
of congestion or component failure, ensuring no data is lost.

For the database we opted for MongoDB\cite{mongo}. We decided this was the best option
for the project because of the necessity to have very flexible data structures.
MongoDB stores collections of JSON objects instead of having rigid tables,
allowing for metadata and optional fields to be added dynamically to objects;
a feature that we specifically wanted for log entries. Due to the lack of
relations in MongoDB it is also known to scale very well, and because MongoDB
is JSON based like the rest of our stack, we would avoid data consistency
issues as no conversion would need to be performed between components.


\section{Bee}

Bee is our data-harvesting agent, which is installed on all monitored machines
and forwards log data into Hive. We designed Bee to be as lightweight and easy
to set up as possible. This is because we wanted Bee to interfere with the
normal operation of the machine it was monitoring as little as possible, and
for the devops task of actually getting the agents running to be trivial.

For this reason Bee has only 1 configuration option - the IP of your central
RabbitMQ cluster. After that all other configuration is either automatically
discovered from the host machine (MAC address, hostname, etc), or pulled in
from Hive after an initial handshake over the RabbitMQ exchange. After this,
Hive can request that this particular instance of Bee listens to and
forwards changes to any file on that particular machine.

Bee is based on NodeJS\cite{node}, as we needed it to be multiplatform, and the
event-driven nature of its framework lends itself well to file monitoring,
as well as responding to incoming RabbitMQ messages. It is designed using
a Model-View-Controller (MVC) framework, with a series of actions for
different commands from Hive.


\section{Hive}

Our core, and most complex component - its role is to manage data agents
(Bees), store incoming log data, and provide complex querying against that
data. In order to achieve scalability, Hive is made up of 6 subcomponents, with
a simple framework in place for extending this with extra components should the
user require additional functionality. All components communicate over the
RabbitMQ message exchange.

Given our background with OpenStack (written in Python) and the abundance of
bindings and support for many of the other technologies we wanted to employ,
Python was the obvious language choice for Hive. Python also provided a clean
multiprocessing library which was crucial for a number of subcomponents. We
chose to implement a common MVC stack for all our components, along with a JSON
protocol that would be shared across the whole stack. This allowed us to
rapidly prototype and add new components to Hive as all the communication
management and database access code were broken out into a reusable library.


\section{Hive Subcomponents} \subsection{Common}

Though not technically a subcomponent in itself - Common is the base layer
underlying almost all of the hive subcomponents. Any code that was duplicated
and was reusable was moved into Common to maintain the DRY (Don’t Repeat
Yourself) programming style. Common therefore became the home for our MVC
stack, including the Parent classes associated with writing controllers and
routers. Common also includes drivers for generic database access, meaning that
the underlying database technology could be changed fairly trivially. The
current model uses PyMongo to hook into MongoDB.

Base pulls all the these parts together into an application that runs but needs
extending to provide any real functionality. It handles the loading of
configuration files and makes sure they are accessible throughout the rest of
the program. It also spawns the worker threads required to listen on the
message bus. Both a subscriber queue and worker queue are listened to in order
to distinguish between RPC (remote procedure call) messages which always
require a response, and purely informative messages - which require action but
no response. Base also sets up and handles all the logging throughout the
program, ensuring that it gets written to an intuitively named file and in a
common format that makes issues easier to find.

\subsection{Agentmanager and Agentmonitor}

These were the first of the Hive subcomponents to be built, the purpose of
Agentmanager and Agentmonitor are to configure and keep track of our Data
Agents.

Agents are, as described earlier in the paper, installed with no configuration,
Agentmanager’s first job is to handshake new Agents entering the environment
and provide them a unique identifier which is attached to data transmissions to
the rest of Hive. After the initial bootstrapping of an Agent, Agentmanager
then receives periodic heartbeats from the Agents. This heartbeat is used by
AgentMonitor to detect the status of an Agent in the case of an unclean
disconnection, error, network congestion, or other problem with the host
machine.

The final task performed by these two components is to provide the Agents with
a series of Data targets, for example in the case of the file watching agent,
this allows a user to remotely set the files that the Agent is monitoring. The
API allows this to be done in bulk, with many files, across many Agents. This
required a more complex response in case that one or more of the Agents failed
to successfully configure, in order to provide the user of the API with enough
information, we collate the individual responses of each Agent configuration
call into a labeled list.

Maintaining the real time aspect of the whole application required the addition
of “events” to Agentmanager and Agentmonitor, whenever a significant change
happens on either component e.g. a new agent handshakes into the environment,
or an agent is flagged as dead, the component where this change happened
published an event object onto a known message exchange, which services, if
they are listening can use to perform any tasks they have which are relevant to
a change in the Agent Manager without the need for polling.

The reason for splitting this component into both Agentmanager and
Agentmonitor, is because the tasks performed by each scale independently.
Agentmanager receives many of messages in the form of heartbeats, and this
grows relative to the number of Agents running. On the other hand Agent
Monitors’ task not increase in complexity nearly as fast, and because of this
it will be far longer before it needs to be scaled out. This configuration of
split services provides the greatest flexibility for host utilisation at large
scale, as you only scale the specific services you need too.

\subsection{Honeycomb}

Honeycomb is our most complicated component as this is where all the storage
and processing of collected logs take place. Honeycomb follows the standard
hive component structure, and uses the MVC stack for inbound requests. It uses
a subscriber queue to digest the stream of data it receives as fast as possible
without having to respond to each log, which would slow the process down.

Honeycomb uses our common model structure to verify the validity of all
received data before it is saved to the database. There is also an additional
step in the Honeycomb save function to perform indexing for every log file
entry. After some research we settled on using PyLucene as a python wrapper for
Lucene to handle indexing. Lucene is an open source plain text analysis and
search tool developed by the Apache Software Foundation. Whenever a log is
saved it is passed to Lucene so that we can utilize their powerful query
language, without having to apply schema to the contents of the log.

As well as storing the logs, Honeycomb also provides the API for searching
them. These searches can be one of two types; a one-off query, or a recurring
query. One-off queries are simply a remote procedure call (RPC) containing the
query. The response is written into a JSON object and passed but to the
requesting process via the message bus. This is great for individual tasks, but
in order to facilitate real time updates we needed to introduce recurring
queries. Recurring queries require a little more setup because they have to run
in a separate thread to the rest of Honeycomb. Since Honeycomb has to remain
stateless for scalability, it should not maintain any reference to these
threads - as this would mean that a second instance of Honeycomb wouldn’t be
able to interact with this query.

In order to solve this issue, we use the message bus again - spawning a wrapper
process which provides an API to the thread in the background, and stores the
key to communicate with that wrapper process in the database. The recurring
query thread isn’t too complex - simply running the lucene query then waiting a
second before running it again. The thread also maintains a list of any
previously discovered log IDs, and will only output if there has been a change,
or if it has been asked to send the next set of results. Output for these
background queries is pushed onto a uniquely named fanout exchange on the
message bus. This allows multiple processeses to receive copies of the output
from a single query, and also ensures that if no one is listening the output is
simply thrown away - instead of clogging up a queue on the message bus.

\subsection{Pheromone}

Using the recurring query API in Honeycomb, Pheromone places a layer of
“intelligence” on top of the query to filter out specific results and trigger
alerts based on user-defined conditions. Alerts are user-defined through the
API Pheromone provides. These alerts take whatever parameters they need to test
for the case on the data, and a custom message that is included when the Alert
is transmitted. Pheromone currently only has the one type of alert test - it
will fire an alert if a query gets a certain number of matches during a given
timescale. However it is not limited to just this one alert type, as the
design patterns used for the “alerters” allows for easy addition of new types
in the future.

Pheromone uses python's multiprocessing library to start background tasks
similar to how Honeycomb’s recurring query tasks operate. When a request comes
in Pheromone starts a new worker for that alert in the background, however
unlike the Honeycomb workers there is no unique output exchange, as these
alerts are always pushed onto a known message bus routing key. This allows
other services, for example Sting, to listen for them and perform all necessary
tasks they have relative to the Alert.

\subsection{Sting}

Pheromone alerts are a very powerful tool, but the user is not always going to
be sitting in front of their computer when an alert is triggered. In fact, the
very nature of user-defined alerts generally makes the times when they will
occur rare and/or unpredictable, and so we must be able to inform the user of
an alert as reliably as possible. By building a native iOS app as described
later in this paper, we are able to take advantage of the Apple Push
Notification Service (APNS). APNS is a service made available by Apple to all
iOS application developers for sending small text-based messages to iOS devices
using your mobile application. In many ways this is similar to the standard
Short Message Service (SMS), but with the key differences that the message is
sent specifically to the application on the device, and more importantly - is
free. Many third-party services exist for sending automatically triggered SMS
messages (for example, Twilio), but these services charge several pence per
message, which could become very expensive in a large system with many users
and alert conditions.

When Pheromone pushes an alert onto the message bus, Sting picks it up and
looks up the user to which the alert is registered. For every iOS device
registered to that user (multiple iPhones, iPads, etc.) Sting will create an
APNS push notification using the alert text specified by the user and the
Device ID associated with the device. The messages are then sent using the
APNS API - typically taking up to 5 seconds to arrive. This system ensures
that a user will always receive potentially critical updates about the state
of their system as and when they happen, giving them the time to act on them
before it may be too late.


\section{Queen}

Our front-end component, also built on the NodeJS framework pulls all of the
components of Hive together to provide a real-time, visually rich experience.

We used the Express\cite{express} library in NodeJS to build an MVC framework for the UI
to
keep in line with our design style throughout the rest of the project. This has
lead to a clean, modular codebase that has very little duplication of code.
The most important decision designing Queen was to ensure that all data
transferred to the front-end would happen in real-time. In order to enable this
we took advantage of the socket.io\cite{socket} library to have web-sockets that allow
client(front-end) to server(Queen back end server) communication.

Like the rest of our components Queen uses RabbitMQ to provide inter-component
communication. The flow for communication with other components is described
here
using a user performing a search as an example:

\begin{enumerate}
  \item The search terms are passed from the UI as a JSON object back to
  Queens
  server via a web-socket.
  \item Queen server then passes this data onto the appropriate Hive
  component
  via RabbitMQ, using either a remote procedure call or
  publish/subscribe model
  as appropriate.
  \item The Hive component (in this case Honeycomb) will then
  return the data to
  Queen server.
  \item The exact data that is required for the
  front-end is extracted and
  passed back up to the UI via a web-socket.
\end{enumerate}

If we request data using a publish/subscribe model, then
every time the Queen
server receives data on the subscribe queue it can push
that data back up to the
front-end using a web-socket. This means that the UI
always has all of the
information available to it and the browser will not
freeze as it would if we
were using a RESTful/polling approach to getting new
data to the UI.

Queen enables multiple users to all be using the service
at once. The principle
for multiple users is to store queries/alerts that the
user wants to access next
time they use the system, also to store a list of
Devices associated to the user
to be used by our iOS application. To store this data
persistently Queen also uses
MongoDB.

\graphicspath{{./pics/}}
\includegraphics[width=\textwidth, keepaspectratio]{data.png}

Just as we did with the Bee agents we wanted to keep
Queen configuration as
simple as we could. The configuration options as such
are kept to just the IP
address of the RabbitMQ server that all of the other
components are talking on and
the IP address of the mongo server. By having the IP
address of the mongo server
as a configuration option Queen can either be running
its own mongo server or
use the existing one setup for Hive.

Rich visualisations were another main goal in designing
Queen, due to our
experience using the d3js\cite{d3} visualisation library we
naturally decided to use
this. It has enabled us to use visualisations such as
sparkline graphs to map
event rates in the system and pie charts to compare search
results. From a
human computer interaction (HCI) point of view this gives
the user access to the
same data in a more meaningful form, and that data can be
accessed at a glance
rather than reading copious amounts of text to come to the
same conclusion.


\graphicspath{{./pics/}}
\includegraphics[width=\textwidth, keepaspectratio]{search.png}

Due to the quantity of data that can come back from search
results it was
important not only to visualise this in graph form but
also allow users to
filter the actual log entries live in the browser opposed
to sending of another
more refined query to Hive. In order to do this we put all
data that we desire
to be searchable in a datagrid from the FuelUX\cite{fuelux} library
which provides instant
searching for all data in the datagrid.

Key to being able to graph data, was being able to infer
schema onto our logs and break queries down into fields. For
this we came up with a system whereby a query could be
broken into subqueries, and each of these subqueries could
represent a member of a field. This allowed us to then take
the results, partition them, and graph the various members
against one another. The subqueries are made independently,
and the results collated and returned to the browser. This
is what allows us to really gain value from otherwise
unorganised log data, as you can now start to make
meaningful analysis.



\section{iOS Application}

In the real world a user will not always be sitting in front of their computer
when an important event occurs in the Apiary framework. However, it is likely
that they will be carrying a modern smartphone, which will have a reasonably
consistent WiFi or cellular data connection. Due to our GUI being entirely
web-based a user could access everything they need to know using their
smartphones web browser, but by creating a native application we can enhance the
user experience and provide additional functionality. A small but useful
function is the ability to save the IP address or URL of the particular Queen
instance to hook into, as well as the users login credentials for that instance,
saving a user the time of configuring the application every time they use it.

The greatest argument for a native application is the ability to exploit the
powerful Apple Push Notification System. As described above in the Sting
section, APNS can be used to send the user real-time updates on the status of
their computer system at anytime, as long as they have a cellular/WiFi data
connection. When a user logs into Apiary using the iOS application, their unique
Apple Device ID is registered against their user account to be used by Sting for
alert notifications.


\section{Future Work}

\subsection{Future Agents}

Very early on in our design we decided that our Agents system should be
flexible enough to accommodate a variety of agent types, as not all schemaless
text data comes from text log files. We had the following Agents planned, but
due to time constraints were not able to implement them, and felt that our time
would be better spent focusing on processing techniques and UI.

First of all, coming from our background with OpenStack, we would have liked to
have built an Agent that was integrated with OpenStack Celiometer, the
monitoring and metrics component of OpenStack. This would have worked in much
the same way, but rather than have the user define a file, they would need to
define a Celiometer Meter to start watching. After this processing would have
worked in exactly the same way.

Another consideration was to build an Agent, or extend Bee so that it could tap
into popular logging frameworks. Many applications have now started to move
away from traditional log files, and instead pump data into logging frameworks
which can maintain slightly more structure about a log. Examples include
LogBack, or Log4j.

Furthermore, other open source projects could be extended to work with our
system, such as FluentD, which has a plugin system that would allow us to build
in Apiary support.

\subsection{Hive - Timemachine and Intelligence}

Some future Hive components we considered but were not within the time scale of
the project are Timemachine and Intelligence. Timemachine would be designed to
pull logs from the database according to a query, and then push them onto a
message queue in the same order that they originally sent. The use case for
this would be an application that requires a replicated stream of historical
data for playing back events as they happened. This historical playback could
be used to analyse system faults to determine what went wrong.

Intelligence is much more complex and has potentially many more use cases. The
idea behind Intelligence is to provide an API for doing Map Reduce tasks with
the log data stored in Honeycomb. Software like Hadoop would allow for more in
depth analysis than is currently possible with the Lucene implementation in
Honeycomb, however both have their place according to the type of query you
need to run. The scope of Intelligence could potentially be extended to include
machine learning which we hope would compliment Pheromone with dynamically
generated alerts.


\section{Conclusion}

To conclude, we set out to design and build an end-to-end solution for real-time
monitoring and analysis of log data, this we feel we have achieved successfully.
We had several goals which sculpted our design and thought processes for the
project, and we believe we have met them with the solution we have built. At the
beginning of the project we separated our goals into the six major categories
that we felt were most important for a system of this type. Addressing them
individually, we can see how our design decisions helped us achieve each one.

\subsection{Scalability}
Splitting the Hive into several components communicating via RabbitMQ has given
us industry-proven levels of scalability at zero cost and with very little
implementation overhead. The use of MongoDB has also given us a database backend
which has been tested to destruction in all corners of the tech industry.
\subsection{Real Time}
RabbitMQ and web sockets have been used to create truly real time communication
between the many components of Apiary and the web front end. Expensive polling
loops are not required to keep the user up to date with the state of their
system.
\subsection{Simple Configuration}
Designing components such as the Bee to self-configure based on data passed from
the central Hive has made deploying Apiary across a distributed system a much
more straightforward task than many open source projects. The inclusion of
installation scripts has also contributed to this goal significantly.
\subsection{Alert System}
Exploiting the real time nature of our message bus infrastructure and the
recurring query API exposed by Honeycomb has allowed us to create the Pheromone
and Sting components. These components work together with the iOS application to
provide the user with real time feedback regardless of their location.
\subsection{Powerful Query System}
By exploiting the advanced text search language that Lucene provides and
combining this with fielding in Queen we have built a system that allows the
user to filter, and infer schema onto otherwise unorganised data. Lucene's query
language is expansive, and we ourselves are yet to fully explore all of the
features it provides. Data can be searched and fielded on a wide range of
parameters, from content, tag, hostname, log time, and many more.
\subsection{User Friendly UI}
Leveraging the power of our query system, our user interface allows the use to
create rich data visualisations from query results, allowing them to make
valuable and meaningful analysis. Our interface is web based, real-time, and
easy to use.

Following the points above, we feel that we have have built a valuable tool,
that allows the user to gain real value from data that would otherwise be locked
away, and difficult to processing. Compared to other projects in the market
today we still sit at a relatively primitive stage, simply due to resources,
however we feel that our implementation brings a number of novel design
decisions that give us an edge in certain applications. No other project, for
example employs our system of using message based communication across the whole
stack, which brings a myriad of benefits. It allows us to do real time,  easy
reliable scaling, and deal with heavy loads without scaling. Our fielding system
is also unique, as other systems have opted for expensive, log analysis
algorithms. Our system allows ultimate flexibility, and keeps overheads down.

If we were to start this project again, or were to give guidance to others
untaking similar projects, we would probably try to avoid using NodeJS in
certain parts of our stack. This is simply to to NodeJSs infancy, we found in
a number of cases that it was difficult to debug, due to the lack of tools,
and lack of predictability in the JavaScript specification. Also, we would
spend a little more time planning the communication protocols, as we had to
make a number of changes, at one point switching from XML to JSON. This took a
significant amount time to refactor, and could have been avoided with more
consideration.

Finally, given more more resources, we would have like to have done some
extremely large scale testing. As it stands we were only able to distribute
instances across the machines that we personally owned, which allowed to to
confirm that scaling worked, but did not demonstrate how far it could be
push, in our testing each component was only instanced no more than 3 or 4
times.




% conference papers do not normally have an appendix


% use section* for acknowledgement
\section*{Acknowledgment}
The authors would like to thank the following people for their assistance during the project:
Ian Utting - http://www.cs.kent.ac.uk/people/staff/iau
Debojji dutta - Cisco Systems, San Jose, CA

% trigger a \newpage just before the given reference
% number - used to balance the columns on the last page
% adjust value as needed - may need to be readjusted if
% the document is modified later
%\IEEEtriggeratref{8}
% The "triggered" command can be changed if desired:
%\IEEEtriggercmd{\enlargethispage{-5in}}

% references section

% can use a bibliography generated by BibTeX as a .bbl file
% BibTeX documentation can be easily obtained at:
% http://www.ctan.org/tex-archive/biblio/bibtex/contrib/doc/
% The IEEEtran BibTeX style support page is at:
% http://www.michaelshell.org/tex/ieeetran/bibtex/
%\bibliographystyle{IEEEtran}
% argument is your BibTeX string definitions and bibliography database(s)
%\bibliography{IEEEabrv,../bib/paper}
%
% <OR> manually copy in the resultant .bbl file
% set second argument of \begin to the number of references
% (used to reserve space for the reference number labels box)
\begin{thebibliography}{1}

  \bibitem{IEEEhowto:kopka}
    H.~Kopka and P.~W. Daly, \emph{A Guide to \LaTeX}, 3rd~ed.\hskip 1em plus
    0.5em minus 0.4em\relax Harlow, England: Addison-Wesley, 1999.

\end{thebibliography}

% that's all folks
\end{document}
